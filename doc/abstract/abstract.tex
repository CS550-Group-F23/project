\documentclass{article}
\usepackage{graphicx} % Required for inserting images

\title{CS550 Project Abstract}
\author{Julien de Castelnau \\ julien.decastelnau@epfl.ch \and Cemalettin Cem Belentepe\\cemalettin.belentepe@epfl.ch}
\date{November 2023}

\begin{document}

\maketitle

Systolic arrays are digital circuits composed of many simple interconnected processing elements, which compute individual operations using data from their neighbors to form a larger result. As compared to traditional architectures (CPU, GPU, etc.), systolic arrays minimize memory traffic by passing intermediate results within the device instead of using a main memory. As a result, they have been highly successful in accelerating parallel digital signal processing algorithms such as matrix multiplication and Fast Fourier Transform (FFT) \cite{540529}. Some notable designs are Google's TPU \cite{46078}, and Nvidia's Tensor Cores \cite{9361255}, which both use systolic architectures to accelerate matrix multiplication for deep learning applications. 

However, despite the increasing complexity of these designs - with Google's latest TPU reaching 22 billion transistors \cite{jouppi2023tpu}, similar to that of a high end CPU - little research has been devoted to the verification of systolic arrays. In our case study, we intend to build a simple but extensible hardware accelerator based on a systolic array, and apply the inductive techniques outlined in \cite{226555} to prove its functionality. We will use the Stainless framework to formally describe the semantics of the hardware, and create tools to compile our Stainless specification to synthesizable Verilog HDL.

\vspace{1em}

Paper of choice: \cite{226555}

\bibliographystyle{plain}

\bibliography{main.bib}

\end{document}

